\documentclass[11point]{article}
% set the margins
\usepackage[left=25mm,right=25mm,top=20mm,bottom=20mm]{geometry}
% for displaying images
\usepackage{graphicx}
% for equation support, e.g. eqref
\usepackage{amsmath}
% for table support, e.g. toprule
\usepackage{booktabs}
% for multiple authors as a block
\usepackage{authblk}

% I redefine vectors to be bold
\renewcommand{\vec}[1]{\boldsymbol{\mathbf{#1}}}

% I can define new macros for things I use a lot
\newcommand{\datapoint}{\vec{x}}
\newcommand{\weights}{\vec{w}}

\DeclareMathOperator{\argmax}{argmax}
\twocolumn

\begin{document}

\title{INST0060 Investigation report:\\Hospital Queuing problem}
\author[*]{Group T1}
\affil[*]{Dept. of Information Studies}
\affil[**]{Dept. of Physics \& Astronomy}
\affil[ ]{University College London, WC1E 6BT}
\affil[ ]{\textit {\{email1,email2\}@ucl.ac.uk}}
\date{\today}

\maketitle

\begin{abstract}
The hospital doesn't provide appointment services and functions as a walk-in centre.In the queuing problem, doctors are assumed to be specialized in three levels: urgent, non-urgent, and critical. Each type of doctor has an associated queue, and there can be any number of doctors for each type. When patients arrive at the hospital, they shall be treated according to their needs.The report would try to simulate a hospital and explore different reinforcement learning algorithms to optimize its policy.
\end{abstract}

\section{Introduction}
\label{sec:intro}

Reinforcement Learning is supposed to act within an environment to achieve goals.

Here is a simple equation
% note this uses the equation environment
\begin{equation}
    \label{eqn:simple_equation}
    \alpha = \sqrt{ \beta }
\end{equation}

I can refer to my equation by saying look at Equation \eqref{eqn:simple_equation}.

Here is a multi-line equation:
% note this uses the align environment
\begin{align}
% \input and \weights in this equation are defined using a macro 
% see the \newcommand lines at the top
f(\datapoint; \weights)
% & helps neatly align equations over multiple lines 
& = \sigma(\weights^T\datapoint)
\label{eqn:logistic_regresion_decision_function} % label
\\ % line break
& = \frac{1}{1+e^{-\weights^T\datapoint}}
\notag % don't give this line an equation number
\end{align}

I can point people to literature like this \cite{Mikolov2013} (a conference paper) or this \cite{Fawcett2006} (a journal paper), or even this \cite[Sec. 4.1]{Bishop2006} (a book, with section indicated). See the file \texttt{report.bib} for how these bibliographic entries are defined.

%It helps to have a convention to the font used for particular mathematical objects. A common convention is: for simple values use lower case latin characters $s$ (this might include observed values in probabilistic work); for sets use caligraphic upper case $\mathcal{S}$; for vectors of values use bold latin $\vec{s}$; for matrices use upper case latin $S$; and for parameters (including parameters in probabilistic models) use lower case greek letters (scalar), $\sigma$, bold greek (vectors), $\vec{\sigma}$, and possibly upper-case greek for matrices of parameters $\Sigma$. However, it is up to you to develop your own style.

\subsection{Subsection Heading}
Write your subsection text here. Here I am referencing Section \ref{sec:intro}. Figures and tables can be inserted and referred to, e.g. Figure \ref{fig:simulation_figure} and Table \ref{tbl:results_table}.

\begin{figure}
    \centering
    \includegraphics[width=3.0in]{figures/myfigure.jpg}
    \caption{Simulation Results}
    \label{fig:simulation_figure}
\end{figure}

\begin{table}[htb]
  \centering
  \begin{tabular}{lcc}
  \toprule
  & Interested & Understand \\
  \midrule
  Linear Regression & N & Y \\
  Neural Networks & Y & N \\
  \bottomrule
  \end{tabular}
  \caption{Some results as a table.}
  \label{tbl:results_table}
\end{table}

\subsection{Quoting and Emphasis}
When you quote or emphasise things, you have a few options. I tend to put things in italics, e.g. Luke told me, \emph{Put it in italics!}. If you prefer to put it in quote-marks then do so, but make sure you use left and right quotes properly, e.g. Luke told me, ``Use left quotes to open, and right quotes to close!''.


%\section{Background}
\section{Methods}
This study started by doing research about MDP, which train algorithms that could be relevant for this kind of problem and if there were any present models or techniques in the area of hospital queuing. Thereafter, when the training algorithm was chosen and the data was collected, it was possible to test the performance of it.
\section{Results}
\section{Discussion}
\section{Conclusion}


\section*{Declaration}
This document represents the group report submission from the named authors for the project assignment of module: Foundations of Machine Learning and Data Science (INST0060), 2019-20. In submitting this document, the authors certify that all submissions for this project are a fair representation of their own work and satisfy the UCL regulations on  plagiarism.

% to put references in
\bibliography{report}
% define the bibliography style
\bibliographystyle{plain}

\end{document}
